\documentclass[a4paper,11pt]{article}
\usepackage[T1]{fontenc}
\usepackage[latin1]{inputenc}
\usepackage{graphicx}
% \usepackage{algorithmic}
% \usepackage{algorithm}
\usepackage[round]{natbib}
\usepackage{url}
\usepackage{booktabs}
\usepackage{dcolumn}
\usepackage{amsmath}
\usepackage{bm}
\usepackage{lmodern}
\usepackage{microtype}
\usepackage[english,spanish,basque]{babel}
\selectlanguage{spanish}
\usepackage{ifthen}
% \usepackage{afterpage}
\usepackage{tikz}
\usetikzlibrary{shapes,arrows,shadows,babel}

\input  variantes.tex


\def\iS{\selectlanguage{spanish}}%
\def\iB{\selectlanguage{basque}}%
\def\iE{\selectlanguage{english}}%

\DeclareGraphicsRule{eps.gz}{epe}{eps.bb}{`gzip -d #1}

% \setkeys{Gin}{width=0.9\textwidth}

\title{Documentaci�n para \texttt{EA3examen}}
% \author{F. Tusell}
\date{23 marzo 2016}

\begin{document}
\maketitle

\section{Introducci�n}

\texttt{EA3examen} es un conjunto de plantillas, macros en \LaTeX{} y
funciones en \textsc{R} y en \textsc{ELISP} para Emacs que buscan facilitar la
producci�n de ex�menes multiling�es con m�ltiples modelos en cada
idioma. Pueden utilizarse s�lo partes, como las definiciones en
\LaTeX, que pueden emplearse en cualquier documento, examen o no, en uno o en
varios idiomas.

La producci�n de ex�menes hace uso del paquete \texttt{examdesign}.


\section{Modus operandi}


\section{Macrodefiniciones}

\subsection{Locuciones frecuentes}

Son expresiones en modo texto que se insertan a veces en una f�rmula o
entre f�rmulas, y constituyen los �nicos elementos que requerir�an
traducci�n. El uso de estos macros permite utilizar una �nica
expresi�n para todos los idiomas. Es tambi�n frecuente la opci�n
\verb|\af| en las preguntas de elecci�n m�ltiple, para
proporcionar una respuesta correcta cuando ninguna m�s lo es.

\begin{table}[htb]
\centering
\caption{Locuciones frecuentes}
\bigskip
\small\setlength{\tabcolsep}{-32pt}
\begin{tabular}{@{}r@{}ccc}
\toprule
 & \multicolumn{3}{c}{\textbf{Variantes idiom�ticas}}  \\[1mm]
\textbf{Macro} & \textbf{Espa�ol} & \textbf{Euskera} & \textbf{Ingl�s} \\
\midrule
\verb|\Y|       & \iS\Y     & \iB \Y     & \iE \Y     \\
\verb|\Sesgo|   & \iS\Sesgo & \iB \Sesgo & \iE \Sesgo \\
\verb|\Var|     & \iS\Var   & \iB \Var   & \iE \Var   \\[1mm]
\verb|\af|      & \iS\parbox[c]{1.25in}{\af}  & \iB  \parbox[c]{1.25in}{\af}  & \iE \parbox[c]{1.25in}{\af}   \\
\bottomrule
\end{tabular}
\normalsize
\end{table}

\section{Notaci�n matem�tica}

\subsection{Constantes}

\begin{table}[htb]
\centering
\caption{Constantes frecuentes}
\bigskip
\small\setlength{\tabcolsep}{-2pt}
\begin{tabular}{@{}r@{}ccc}
\toprule
 & \multicolumn{3}{c}{\textbf{Variantes idiom�ticas}}  \\[1mm]
\textbf{Macro} & \textbf{Espa�ol} & \textbf{Euskera} & \textbf{Ingl�s} \\
\midrule
\verb|\oh|       & \iS\oh  & \iB\oh   & \iE\oh    \\[1mm]
\verb|\rdpi|       & \iS\rdpi  & \iB\rdpi   & \iE\rdpi    \\
\bottomrule
\end{tabular}
\normalsize
\end{table}


\subsection{Estimadores}
\label{estimadores}

\begin{table}[htb]
\centering
\caption{Estimadores frecuentes}
\bigskip
\small\setlength{\tabcolsep}{-2pt}
\begin{tabular}{@{}r@{}ccc}
\toprule
 & \multicolumn{3}{c}{\textbf{Variantes idiom�ticas}}  \\[1mm]
\textbf{Macro} & \textbf{Espa�ol} & \textbf{Euskera} & \textbf{Ingl�s} \\
\midrule
\verb|\emv{\theta}|   & \iS\emv{\theta}  & \iB\emv{\theta}   & \iE\emv{\theta}   \\
\verb|\emom{\theta}|  & \iS\emom{\theta} & \iB\emom{\theta}  & \iE\emom{\theta}   \\
\bottomrule
\end{tabular}
\normalsize
\end{table}


\subsection{Funciones y expresiones}

\begin{table}[htb]
\centering
\caption{Funciones y expresiones frecuentes}
\bigskip
\small\setlength{\tabcolsep}{-12pt}
\begin{tabular}{@{}r@{}ccc}
\toprule
 & \multicolumn{3}{c}{\textbf{Variantes idiom�ticas}}  \\[1mm]
\textbf{Macro} & \textbf{Espa�ol} & \textbf{Euskera} & \textbf{Ingl�s} \\
\midrule
\verb|\f{x}|       & \iS\f{x}  & \iB\f{x}   & \iE\f{x}   \\
\verb|\F{x}|       & \iS\F{x}  & \iB\F{x}   & \iE\F{x}   \\
\verb|\char{x}|    & \iS\char{x}  & \iB\char{x}   & \iE\char{x}   \\
\verb|\momg{x}|    & \iS\momg{x}  & \iB\momg{x}   & \iE\momg{x}   \\
\verb|\ecm{\hat\theta}| & \iS\ecm{\hat{\theta}} & \iB\ecm{\hat{\theta}} & \iE\ecm{\hat{\theta}} \\
\bottomrule
\end{tabular}
\normalsize
\end{table}

\paragraph{Ejemplos de uso.} Como argumento de \verb|ecm| se pueden
insertar cualesquiera estimadores de los definidos en la
subsecci�n~\ref{estimadores}. Por ejemplo, podemos escribir:
\bigskip

\selectlanguage{spanish}
\noindent\begin{tabular}[l]{lp{0.5cm}l}
El \verb|\ecm{\emv{\theta}}| es$\ldots$  &  & El \ecm{\emv{\theta}}
es$\ldots$ \\
\verb|\ecm{\emv{\pi}} = \ecm{\emom{\pi}}| & & $\ecm{\emv{\pi}}
= \ecm{\emom{\pi}}$ \\
\verb|\ecm{\hat\pi} = \Sesgo(\hat\pi) +| \\
\qquad\qquad \verb|\Var(\hat\pi)| & & $\ecm{\hat\pi} =
  \Sesgo(\hat\pi) + \Var(\hat\pi)$ \\
\end{tabular}



\subsection{L�mites}

Un macro general, \verb|\anylim{texto}| sit�a ``texto'' sobre una flecha
hacia la derecha. Hay especializaciones para l�mites en distribuci�n,
probabilidad y casi seguros (o ``con probabilidad unitaria'').


\begin{table}[htb]
\centering
\caption{L�mites probabilisticos}
\bigskip
\small\setlength{\tabcolsep}{-2pt}
\begin{tabular}{@{}r@{}ccc}
\toprule
 & \multicolumn{3}{c}{\textbf{Variantes idiom�ticas}}  \\[1mm]
\textbf{Macro} & \textbf{Espa�ol} & \textbf{Euskera} & \textbf{Ingl�s} \\
\midrule
\verb|\anylim{texto}|       & \iS\anylim{texto}  & \iB\anylim{texto}   & \iE\anylim{texto}    \\
\verb|\dlim|    & \iS\dlim  & \iB \dlim  & \iE \dlim \\
\verb|\plim|    & \iS\plim  & \iB \plim  & \iE \plim \\
\verb|\cslim|   & \iS\cslim & \iB \cslim & \iE \cslim \\
\bottomrule
\end{tabular}
\normalsize
\end{table}

\pgfdeclarelayer{background}
\pgfdeclarelayer{foreground}
\pgfsetlayers{background,main,foreground}

\tikzstyle{input}=[draw, fill=red!20, text width=5em,
    text centered, minimum height=3em, rounded corners, drop shadow]
\tikzstyle{output}=[draw, fill=blue!20, text width=5em,
    text centered, minimum height=3em, rounded corners, drop shadow]
\tikzstyle{ba} = [input, text width=5em, fill=blue!20,
    minimum height=3em, rounded corners, drop shadow]
\def\blockdist{3.3}
\def\edgedist{2.5}
\def\interblock{2.4}

\begin{figure}
\caption{Proceso de generaci�n de documentos con \texttt{EA3examen}}
\bigskip\bigskip

\begin{tikzpicture}
\node (ba) [ba]  {\LaTeX\\ \texttt{.tex}};
\path (ba.west)+(-\interblock, 1.5) node (rnw) [input] {Sweave\\ \texttt{.Rnw}};
\path (ba.west)+(-\interblock,-1.5) node (tex) [input] {\LaTeX\\ \texttt{.tex}};
\path (ba.east)+( \interblock, 1.5) node (EN) [output] {EN \\ \texttt{.tex}};
\path (ba.east)+( \interblock, 0.0) node (ES) [output] {ES \\ \texttt{.tex}};
\path (ba.east)+( \interblock,-1.5) node (EU) [output] {EU \\ \texttt{.tex}};
\path (EN.east)+( \interblock, 0.0) node (ENpdf) [output] {EN \\ \texttt{.pdf}};
\path (ES.east)+( \interblock, 0.0) node (ESpdf) [output] {ES \\ \texttt{.pdf}};
\path (EU.east)+( \interblock, 0.0) node (EUpdf) [output] {EU \\ \texttt{.pdf}};


\path [draw, ->] (rnw.east) -- node [above] {}  (ba.175) ;
\path [draw, ->] (tex.east) -- node [above] {}  (ba.185);
\path [draw, ->] (ba.east)  -- node [left] {}  (EN.180);
\path [draw, ->] (ba.east)  -- node [above] {}  (ES.180);
\path [draw, ->] (ba.east)  -- node [above] {}  (EU.180);
\path [draw, ->] (EN.east)  -- node [above] {}  (ENpdf.180);
\path [draw, ->] (ES.east)  -- node [above] {}  (ESpdf.180);
\path [draw, ->] (EU.east)  -- node [above] {}  (EUpdf.180);


%     \path (ba.south) +(1.5,-\blockdist) node (asrs) {Proceso generaci�n ex�menes};
    %  \begin{pgfonlayer}{background}
    %      \path (rnw.west |- rnw.north)+(-0.5,0.3) node (a) {};
    %      \path (ba.south -| ba.east)+(+0.5,-0.3) node (b) {};
    %      \path (EUpdf.east |- asrs.east)+(+0.5,-0.5) node (c) {};

    %     \path[fill=yellow!20,rounded corners, draw=black!50] %  dashed]
    %         (a) rectangle (c);
    %     \path (rnw.north west)+(-0.2,0.2) node (a) {};
    % \end{pgfonlayer}

\end{tikzpicture}
\end{figure}



\end{document}
