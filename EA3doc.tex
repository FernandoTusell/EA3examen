\documentclass[a4paper,11pt]{article}
\usepackage[T1]{fontenc}
\usepackage[latin1]{inputenc}
\usepackage{graphicx}
% \usepackage{algorithmic}
% \usepackage{algorithm}
\usepackage[round]{natbib}
\usepackage{url}
\usepackage{booktabs}
\usepackage{dcolumn}
\usepackage{amsmath}
\usepackage{lmodern}
\usepackage{microtype}
\usepackage[english,spanish,basque]{babel}
\selectlanguage{spanish}
\usepackage{ifthen}
\usepackage{afterpage}

\input  variantes.tex
\def\iS{\selectlanguage{spanish}}%
\def\iB{\selectlanguage{basque}}%
\def\iE{\selectlanguage{english}}%

\DeclareGraphicsRule{eps.gz}{epe}{eps.bb}{`gzip -d #1}

% \setkeys{Gin}{width=0.9\textwidth}

\title{Documentaci�n para \texttt{EA3examen}}
\author{F. Tusell}
\date{23 marzo 2016}

\begin{document}
\maketitle

\section{Introducci�n}

\texttt{EA3examen} es un conjunto de plantillas, macros en \LaTeX y
funciones en \textsc{ELisp} para Emacs que buscan facilitar la
producci�n de ex�menes multiling�es con m�ltiples modelos en cada
idioma. Pueden utilizarse s�lo partes, como las definiciones en
\LaTeX, que pueden emplearse en cualquier documento, examen o no, en uno o en
varios idiomas.

La producci�n de ex�menes hace uso del paquete \texttt{examdesign}.

\section{Macrodefiniciones}

\subsection{Locuciones frecuentes}

Son expresiones en modo texto que se insertan a veces en una f�rmula o
entre f�rmulas, y constituyen los �nicos elementos que requerir�an
traducci�n. El uso de estos macros permite utilizar una �nica
expresi�n para todos los idiomas. Es tambi�n frecuente la opci�n
\verb|\af| en las preguntas de elecci�n m�ltiple, para
proporcionar una respuesta correcta cuando ninguna m�s lo es.

\begin{table}[htb]
\centering
\caption{Locuciones frecuentes}
\bigskip
\small\setlength{\tabcolsep}{-22pt}
\begin{tabular}{@{}r@{}ccc}
\toprule
 & \multicolumn{3}{c}{\textbf{Variantes idiom�ticas}}  \\[1mm]
\textbf{Macro} & \textbf{Espa�ol} & \textbf{Euskera} & \textbf{Ingl�s} \\
\midrule
\verb|\Y|       & \iS\Y     & \iB \Y     & \iE \Y     \\
\verb|\Sesgo|   & \iS\Sesgo & \iB \Sesgo & \iE \Sesgo \\
\verb|\Var|     & \iS\Var   & \iB \Var   & \iE \Var   \\[1mm]
\verb|\af|      & \iS\parbox[c]{1.25in}{\af}  & \iB  \parbox[c]{1.25in}{\af}  & \iE \parbox[c]{1.25in}{\af}   \\
\bottomrule
\end{tabular}
\normalsize
\end{table}

\section{Notaci�n matem�tica}

\subsection{L�mites}

Un macro general, \verb|\anylim{texto}| sit�a ``texto'' sobre una flecha
hacia la derecha. Hay especializaciones para l�mites en distribuci�n,
probabilidad y casi seguros (o ``con probabilidad unitaria'').


\begin{table}[htb]
\centering
\caption{L�mites probabilisticos}
\bigskip
\small\setlength{\tabcolsep}{-2pt}
\begin{tabular}{@{}r@{}ccc}
\toprule
 & \multicolumn{3}{c}{\textbf{Variantes idiom�ticas}}  \\[1mm]
\textbf{Macro} & \textbf{Espa�ol} & \textbf{Euskera} & \textbf{Ingl�s} \\
\midrule
\verb|\anylim{texto}|       & \iS\anylim{texto}  & \iB\anylim{texto}   & \iE\anylim{texto}    \\
\verb|\dlim|    & \iS\dlim  & \iB \dlim  & \iE \dlim \\
\verb|\plim|    & \iS\plim  & \iB \plim  & \iE \plim \\
\verb|\cslim|   & \iS\cslim & \iB \cslim & \iE \cslim \\
\bottomrule
\end{tabular}
\normalsize
\end{table}

\end{document}
